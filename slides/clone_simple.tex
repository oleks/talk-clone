\begin{frame}[fragile]

\frametitle{\texttt{clone(}``\texttt{2}''\texttt{)}}

\footnotesize

\lstinputlisting[firstline=14]{../src/clone_simple.c}

\end{frame}



% \begin{frame}[fragile]
% 
% \frametitle{\texttt{man} ``2'' \texttt{clone}}
% 
% \footnotesize
% 
% The low byte of \texttt{flags} contains the number of the termination signal
% sent to the parent when the child dies.  If this signal is specified as
% anything other than \texttt{SIGCHLD}, then the parent process must specify the
% \texttt{\_\_WALL} or \texttt{\_\_WCLONE} options when  waiting  for  the child
% with \texttt{wait(2)}.  If no signal is specified, then the parent process is
% not signaled when the child terminates.
% 
% \end{frame}

\begin{frame}[fragile]

\frametitle{\texttt{man} ``2'' \texttt{clone}}

\footnotesize

\begin{manblock}

\textbf{C library/kernel differences}

\medskip

The raw \texttt{clone()} system call corresponds more closely to
\texttt{fork(2)} in that execution  in the  child continues from the point of
the call.  As such, the \texttt{fn} and \texttt{arg} arguments of the
\texttt{clone()} wrapper function are omitted.  Furthermore, the  argument
order  changes.  The raw system call interface on x86 and many other
architectures is roughly:

\begin{lstlisting}[language=c]
long clone(unsigned long flags, void *child_stack,
           void *ptid, void *ctid,
           struct pt_regs *regs);
\end{lstlisting}

Another  difference  for  the  raw system call is that the
\texttt{child\_stack} argument may be zero, in which case copy-on-write
semantics ensure that the child gets separate copies of  stack  pages  when
either  process modifies the stack.  In this case, for correct operation, the
\texttt{CLONE\_VM} option should not be specified.

\medskip

For some architectures, the order of the arguments for the system  call differs
from that  shown  above.

\end{manblock}

\end{frame}

\begin{frame}

\frametitle{\texttt{man} ``2'' \texttt{clone}}

\end{frame}

\begin{frame}

\frametitle{Control groups}

\end{frame}
