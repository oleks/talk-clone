\begin{frame}[fragile]

\frametitle{\texttt{man} ``2'' \texttt{clone}}

\footnotesize

\begin{manblock}

\begin{lstlisting}
int clone(int (*fn)(void *), void *child_stack,
          int flags, void *arg, ...
\end{lstlisting}

\texttt{clone()} creates a new process, in a manner similar to \texttt{fork(2)}.

\medskip

($\ldots$)

\medskip

Unlike \texttt{fork(2)}, \texttt{clone()} allows the child process to
share parts of its execution context with the calling process, such as the
memory space, the table of file descriptors, and the table of signal handlers.

($\ldots$)

\medskip

When the child process is created with \texttt{clone()}, \\ it  executes
the function \texttt{fn(arg)}.

($\ldots$)

\medskip

When the \texttt{fn(arg)} function application returns,  the  child process
terminates. The integer returned by \texttt{fn} is the exit code for the child
process.

% The child process may also terminate explicitly by calling
% \texttt{exit(2)} or after receiving a fatal signal.

\end{manblock}

\end{frame}
