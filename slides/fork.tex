\begin{frame}[fragile]

\frametitle{\texttt{fork(}``\texttt{2}''\texttt{)}}

\footnotesize

\lstinputlisting[firstline=10]{../src/fork.c}

\end{frame}


\begin{frame}[fragile]

\frametitle{\texttt{make fork-trace} (i.e., \texttt{strace -f ./fork.bin})}

\footnotesize

\input{gen-fork-trace}

\end{frame}


\begin{frame}[fragile]

\frametitle{\texttt{man} ``2'' \texttt{fork}}

\footnotesize

\begin{manblock}

\textbf{C library/kernel differences}

\medskip

Since  version  2.3.3,  rather  than invoking the kernel's \texttt{fork()}
system call, the glibc \texttt{fork()} wrapper that is provided as part of the
NPTL threading implementation invokes \texttt{clone(2)} with flags that provide
the same effect as the traditional system call. ($\ldots$)

\end{manblock}

% (A call to \texttt{fork()} is equivalent to a call to \texttt{clone(2)}
% specifying \texttt{flags} as just \texttt{SIGCHLD}.)

\begin{itemize}

\item NPTL stands for Native POSIX Thread Library.

\item POSIX stands for Portable Operating System Interface.

\end{itemize}

\end{frame}
